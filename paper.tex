% ****** Start of file aipsamp.tex ******
%
%   This file is part of the AIP files in the AIP distribution for REVTeX 4.
%   Version 4.1 of REVTeX, October 2009
%
%   Copyright (c) 2009 American Institute of Physics.

% Use this file as a source of example code for your aip document.
% Use the file aiptemplate.tex as a template for your document.
\documentclass[%
 aip,
 jmp,%
 amsmath,amssymb,
%preprint,%
 reprint,%
%author-year,%
%author-numerical,%
]{revtex4-1}

\usepackage{enumitem}
\usepackage{graphicx}% Include figure files
\usepackage{grffile}
\usepackage{dcolumn}% Align table columns on decimal point
\usepackage{bm}% bold math
%\usepackage[mathlines]{lineno}% Enable numbering of text and display math
%\linenumbers\relax % Commence numbering lines
\usepackage{multirow}
\usepackage{color} % for the notes
\usepackage{etex}
\reserveinserts{58}
\usepackage{morefloats}
\usepackage{hyperref}
\usepackage{xcolor}
\hypersetup{
        colorlinks,
        linkcolor={red!50!black},
        citecolor={blue!50!black},
        urlcolor={blue!80!black}
}


\maxdeadcycles=1000

\begin{document}

\preprint{XXXXX (preprint)}

%\title[Evolution of interaction networks]{On the evolution of interaction networks: primitive typology of vertex, prominence of measures and activity statistics}% Force line breaks with \\
%\title[Evolution of interaction networks]{On the evolution of interaction networks: a primitive typology of vertex}% Force line breaks with \\
\title[Data-driven ontology synthesis]{A method for the synthesis of an
ontology from predefined data}% Force line breaks with \\

\author{Renato Fabbri}%
 \homepage{http://ifsc.usp.br/~fabbri/}
 \email{fabbri@usp.br}
  \affiliation{ 
S\~ao Carlos Institute of Physics, University of S\~ao Paulo (IFSC/USP)%\\This line break forced with \textbackslash\textbackslash
}

\author{Osvaldo N. Oliveira Jr.}
  \homepage{www.polimeros.ifsc.usp.br/professors/professor.php?id=4}
  \email{chu@ifsc.usp.br}
 \altaffiliation[Also at ]{IFSC-USP}%Lines break automatically or can be forced with \\


\date{\today}% It is always \today, today,
             %  but any date may be explicitly specified

\begin{abstract}
OWL Ontologies are critical tools to describe taxonomies and the
structure of knowledge.
Most ontologies are created by domain experts even though the data they
arrange is often given by a software system.
This work presents a method for obtaining an ontology from predefined
data.
The resulting structure has the simplest form with an accurate
support for the undelying information.
\end{abstract}

% \pacs{89.75.Fb,05.65.+b,89.65.-s}% PACS, the Physics and Astronomy
\keywords{linked data, semantic web, big data, artificial intelligence}%Use showkeys class option if keyword
\maketitle
                   
\section{Introduction}
Ontologies and data-driven ontologies.

\subsection{Related work}
Ontology creation methods and data-driven methods.

\section{Method}
The method consists of probing the ontological structure in data with
SPARQL queries and post-processing which can be divided in the following steps:
\begin{enumerate}[leftmargin=0cm]
    \item Obtaining all distinct classes with the query:\\
        \texttt{SELECT DISTINCT ?class WHERE \{ ?s a ?class \}}
    \item Obtaining all distinct properties with the query:\\
        \texttt{SELECT DISTINCT ?p WHERE \{ ?s ?p ?o \}}
    \item For each class, get distinct subject classes and predicates where the
        the object is an instance of the class:\\
        \texttt{SELECT DISTINCT ?p ?cs WHERE \{ ?i a <class\_uri> . ?s ?p ?i . ?s
        a ?cs . \}}
    \item For each class, get distinct predicates and object classes or
        datatypes where the subject is an instance of such class:\\
        \texttt{SELECT DISTINCT ?p ?co (datatype(?o) as ?do) WHERE \{ ?i
                a <class\_uri> . ?i ?p ?o . OPTIONAL \{ ?o a ?co . \} \}}
    \item For each property, check if it is functional, i.e. if it
        occurs only once with each subject:\\
        \texttt{SELECT DISTINCT (COUNT(?o) as ?co) WHERE \{ ?s
            <property\_uri> ?o \} GROUP BY ?s}
    \item For each property, find the incident range and domain with the
        queries:\\
        \texttt{SELECT DISTINCT ?co (datatype(?o) as ?do) WHERE \{ ?s
                <property\_uri> ?o . OPTIONAL \{ ?o a ?co . \} \}}
        \texttt{SELECT DISTINCT ?cs WHERE \{ ?s <property\_uri> ?o . ?s a ?cs . \}}
    \item For each instance of each class, get all distinct predicates.
        For each predicate, check if all instances of the class
        hold such relationship (existential restriction):\\
        \texttt{SELECT DISTINCT ?p WHERE \{ ?s a <class\_uri>. ?s ?p ?o
        . \}}\\
        \texttt{SELECT DISTINCT ?s WHERE \{ ?s a <class\_uri> \}}\\
        \texttt{SELECT DISTINCT ?s ?co  (datatype(?o) as ?do) WHERE \{?s
                a <class\_uri>. ?s <property\_uri> ?o . OPTIONAL \{?o a ?co . \}\}}
    \item and if all instances that hold such relationship are instances of the class
        (universal restriction):\\
        \texttt{SELECT DISTINCT ?s WHERE \{ ?s <property\_uri> ?o . \}}
    \item Draw each class, each property and the overall figure.
    \item Make \texttt{rdfs:subClassOf} and \texttt{rdfs:subPropertyOf}
        statements to better organize knowledge and link to third party
        ontologies and data.
    \item
\end{enumerate}

\section{Results and discussion}
Figure
\section{Conclusions}

\begin{acknowledgments}
Renato Fabbri is grateful to CNPq (process: 140860/2013-4,
project 870336/1997-5), United Nations Development Program (PNUD/ONU, contract: 2013/000566; project BRA/12/018)  and 
the Postgraduate Committee of the IFSC/USP.
\end{acknowledgments}


%%%%%%%%%%%%%%%%%%%%%%%%%%%%%%%%%%%%%%%
\appendix
\section{Foo}
Bar

\nocite{*}
\bibliography{paper}% Produces the bibliography via BibTeX.

\end{document}
%
% ****** End of file aipsamp.tex ******



